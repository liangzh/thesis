%\cleardoublepage\phantomsection
%\addcontentsline{toc}{chapter}{Chinese Abstract}
\pagenumbering{roman}

\chapter*{\LARGE  \bfseries {摘\ \ \ \ \ \ 要} }


\qquad 研究原子核及其核子的结构和它们之间的相互作用是核物理研究的主要内容。这一类研究揭示了构成我们可见物质世界中的绝大部分质量组分的基本微观性质。
过去半个多世纪的研究工作表明,原子核内的核子是由叫做夸克的基本粒子通过交换胶子的相互作用被束缚在一起而形成的。
为了理解和描述这一类基本相互作用,量子色动力学(Quantum Chromodynamics,
QCD)在历经半个多世纪的理论与实验工作的交替发展中被逐步建立了起来。通过
QCD框架下的夸克胶子动力学来认识核物质的基本组成被认为是现代核物理研究中的一个重要目标。

QCD中的相互作用基于夸克和胶子所带的色荷。胶子作为相互作用传播子,它本身也带有色荷,这一性质决定了
胶子的自相互作用在核结构的形成中会起到重要的作用。在QCD理论中,与核子发生的碰撞过程通常可以理解为和核子中的
夸克和胶子的一系列基本碰撞过程的非相干叠加,并从中拟合出可普适性地应用于不同碰撞过程的部分子分布函数(Parton Distribution Function,
PDF)以描述夸克和胶子在核子中的密度分布。 过去的实验较好地测定了在核子和轻核中的夸克分布函数,但是对于胶子
的动力学行为还没有能够获得很精确的认识。尤其是对于动量份额比较小(small $x$)的区域,胶子的动力学效应将会
在核子的PDF中占据主导地位。在这一胶子占有数密度比较大、动量份额比较小的动力学区域,胶子的自相互将有可能在核子 的波函数中引起一个被称为胶子饱和的效应。

胶子饱和效应来源于胶子动力学演化中的非线性作用。这一现象在目前的实验中得到了部分的验证,但是由于实验的
精度和模型依赖性的问题,还无法对这一现象的存在以及产生机制给出确定性的结论。未来的电子-重离子对撞机(Electron-Ion Collider, EIC)
将会为解答这一问题提供更多精确的线索。EIC能够提供宽能量范围内的多种核粒子束流。这将为精确 确定胶子行为提供良好的实验条件。

在本文中,我们通过蒙特卡罗模拟的方法,研究了通过在EIC上的双强子关联这一观测量来限制胶子的动力学行为以及找寻
胶子饱和效应的可行性。此外,我们还探讨了一种确定电子-重离子对撞过程中的碰撞几何特性的方法。我们的研究表明,通过
EIC上的双强子关联测量我们能够精确地判断胶子饱和效应是否存在,并给出在胶子饱和区的胶子动力学特性。

\vspace{4mm}

\noindent
\textbf{关键词:} 电子-重离子对撞机(EIC), 电子-重离子对撞(\eA ), 量子色动力学 (QCD),胶子饱和, 色玻璃凝聚态, 双强子关联

