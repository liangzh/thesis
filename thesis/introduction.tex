\chapter{Introduction}
\label{chp:introduction}
You may still remember the times when you were looking into the dark blue sky at
a summer night as a young kid. Your little brain could be stormed by numerous
interesting questions. For example, why are the stars shining in the sky? Why
don't they fall down to the earth? Is there anybody living on the stars? If yes,
do they look the same as us? And so on. Curiosity demands that we ask questions.
As we grow up, it's the same human nature of curiosity that drives us to ask
more deeper questions and come up with reasonable answers to those questions.
Particle physicists are a group of people who are driven by the purest human
curiosity to answer the most ambitious questions with the most organized effort.
The task of particle physics is to understand the question what the world is
made of. Mendeleev's periodic table~\cite{Mende:1869} is one answer to this
question but in a very complicated form. The proliferation and periodic
organization of elements in this table strongly suggests a substructure.
Following the same spirit to explore the unknown world as Columbus who
discovered the new lands, standing on the shoulder of Mendeleev, particle
physicists found their way to the terra incognita rooted deeply in the
fundamental structure of our most immediate world.

Starting from the experimental effort of Rutherford~\cite{Rutherford:1911},
people realized that the element in Mendeleev's table is consisting of a single
type of atom built up of electrons and nuclei in a core like formation. The
atomic nucleus sitting at the core of an atom is composed of more fundamental
particles labeled as protons and neutrons. Later discoveries indicate that these
particles are not alone. They are just two particles in a group called hadrons,
in which all the particles undergo strong interactions. Again, the same argument
repeats that the proliferation of these hadrons is an indication for the
existence of more elementary constituents.

Deep Inelastic Scattering (DIS) is a process that enables us to directly ``see"
the underlying constituents of a particle in a very clean way. To explain the
overwhelming results from DIS and other experiments, the quark parton model is
proposed to understand the microscopic details about the scattering process. It
is developed in this framework that the scattering on the hadronic objects can
be written as an incoherent sum of cross sections from the scatterings on
individual partons multiplying the hadron's parton distribution function (PDF).
More specifically, the partons are referring to the quarks and gluons. Our
current understanding to the basic structure of all the visible matter can be
interpreted within this model. Gluons carry the strong force and
``glues" quarks together into protons, neutrons and, further, all the hard
cores of the atoms making up the matter in our visible universe. The theoretical
and experimental impulse to describe the properties and dynamics of quarks and
gluons has led to the development of quantum chromodynamics (QCD), which is a
cornerstone to understand the emergence of nucleons and nuclei.


Although past experiments were successful in determining the quark behavior in
the nucleon and light nuclei, the gluons that dictate the essential features
of the strong interactions, remain largely unexplored. Of great interest is
especially, when the gluon density grows to a point where the self-interaction
of gluons have become so important that non-linear QCD effects supersede, a
phenomena named saturation. It is a universal behavior of gluons existing in any
hadronic systems ranging from pions, protons to nuclei. To date, it is still not
yet conclusive that such a saturated regime has been discovered at presently
running high energy experimental facilities. This pursuit will be facilitated by
the advent of a proposed high luminosity, high energy electron-ion collider (EIC).


The present thesis is devoted to the phenomenological studies of the dihadron
correlation measurement to be performed at an EIC. In the remainder of this
thesis, I will walk through the general framework of QCD formulated to
understand the successive experimental results paving us the way to the
understanding of the fundamental constituents for our universe in
Chpt.~\ref{chp:theory}. The conceptual formalism of saturation and its
experimental search have been briefed in Chpt.~\ref{chp:saturation}. An
introduction of the EIC project at its realization will be described in
Chpt.~\ref{chp:EIC}. In Chpt.~\ref{chp:MC}, we explained our Monte Carlo
simulation tools used in this study. Based on the current EIC conceptual design,
the we will discuss our simulation work of the dihadron correlation studies at
an EIC in Chpt.~\ref{chp:dihadron}. In the end, we propose a possible geometry
handle in the electron-nucleus collision and explored the prospective
application of this handle in the measurements like dihadron correlation in
Chpt.~\ref{chp:geometry}. A summary is given in Chpt.~\ref{chp:summary}.

