\chapter{Summary and outlook} \label{chp:summary}

In this thesis, dihadron correlation measurement as a probe for the saturation
physics has been studied. The feasibility of performing this measurement at a
future EIC is explored with our Monte Carlo simulation method. We have also
discussed the possibility of carrying out a collision geometry study in \eA\,
which may be applied to the dihadron correlation measurement and providing
additional systematics control.

An analysis accounting for gluon radiation in the calculation of the dihadron
cross section in \eA\ through Sudakov factor within the saturation framework has
been included. Based on the knowledge of this correction one can make a
comparison with the theory calculation and the future EIC data. In order to
understand the sensitivity of this dihadron correlation measurement to gluon
saturation, a non-saturation based model has been developed in our studies based
on the widely used \ep\ Monte Carlo generator PYTHIA interfaced with the nuclear
PDF and the cold nuclear medium energy loss effect. On the other hand, a
measurement of constraining collision geometry for \eA\ has been proposed and
investigated. This measurement can be easily performed by measuring neutral
energy deposition in the ZDC, which is very vital and beneficial for the \eA\ program at an EIC. 

This thesis summarized the many studies performed in the course of investigating
the saturation physics and the interesting results thus obtained. After all, there
are still open questions and ideas for future studies based on the current
results.

It could be very interesting to further understand the nuclear dependence of the
Sudakov factor in dihadron correlations. Currently, it is found that there's no
nuclear medium dependence for this Sudakov factor (parton shower) at leading
log. More sophisticated estimations need the input of experimental data to
constrain higher order effects. The forthcoming \pA\ data may shed some light on
the determination of this nuclear modification. Along with that, we probably
need more robust model independent way to estimate the impact of the collision
geometry control over various observables. This geometry control if achieved
will definitely bring us a new way of thinking about the nuclear effect in the
future \eA\ data.
