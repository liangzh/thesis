\chapter*{\LARGE  \bfseries {致\ \ \ \ \ \ 谢} }
\addcontentsline{toc}{chapter}{Acknowledgment}


%首段空两格
\qquad
五年多的硕博生活即将走到终点。回望走过的日子,有太多的人在其间给予过我很多的帮助。仓促成文之际,只能在这里
对他们简短地表达一下自己的谢意。

首先要由衷地感谢我的导师蔡勖教授多年的关心和指导。他渊博的学识,高屋建瓴的学术见解和认真忘我的工作态度为我树立了
人生的标杆。他为我创造的科研条件让我在硕博阶段的学习中受益颇多。因为他一贯的支持和鼓励也使我有勇气在博士阶段
去尝试一个完全不同的领域。

感谢布鲁克海文国家实验室的Jeong-Hun Lee研究员。我在BNL的三年多时间里,非常感谢他对我生活上的关心和学术上的指导。
他清晰的物理图像和对物理问题的深刻洞察力经常会让我在和他讨论问题的时候有豁然开朗的感觉。他和学生平等相处的态度
以及对学生观点的尊重让我的科研能力得到了充分的锻炼。

感谢殷中宝教授给我介绍了参与EIC项目的机会,并一直督促我在BNL的学习工作。在我论文撰写的过程中,
殷中宝教授不厌其烦地帮我修改论文的细节还提出了很多宝贵的建设性建议,使我能够在相对有限的时间里及时完成博士论文
的撰写工作。他丰富的研究经验和严谨的治学态度给我的研究和生活都带来了很多新的启示。

感谢李炜教授对我学业上的指导。从本科阶段的毕业论文工作到研究生阶段复杂系统的研究,李炜教授的指导为我
整个研究生阶段的工作打下了坚实的基础。他身上所体现出的对科学的热爱和严谨的科研精神将使我受益终身。

感谢肖博文教授在我博士期间的工作中对我的帮助。他在我刚刚接触相关工作的时候就用深入浅出的语言帮我讲解了
整个研究工作的理论基础,并且花了很多时间去回答我不理解的技术性问题,提出了很多新颖而有创见性的意见,给
我的研究工作带了很多重要的新思路。而他勤勉的工作态度,忘我的研究精神更是我在人生道路上的学习榜样。

感谢池丽萍老师和杨纯斌老师在我们的小组讨论中对我学业上的指导。感谢周代翠老师对我学业上的帮助以及在我准备博士答辩期间为我提供的便利条件。感谢粒子所王恩科老师,刘峰老师,
杨亚东老师,侯德富老师,刘复明老师,付菁华老师,许明梅老师在平时学业上对我的教诲。感谢
高燕敏老师,谢晓梅老师,马亚老师,刘海涛老师,刘超老师,葛静老师对我们生活上的关心,为我们提供了良好的学习工作
条件。

感谢复杂系统小组的各位师兄师姐:郭龙,江健,辜姣,惠子,邓为炳,曹燕青,赵婷婷,还有王杜鹃同学和各位师弟师妹:朱月英,
骆增增,章可成,粟柱,赵龙峰,韩继辉,邹以江,褚晓璇,何长洋,张文俊,邓盛峰,刘光环,王冰冰,王可,徐高。感谢你们给了
我一个大家庭般的温暖。

感谢ALICE小组的几位师弟:张永红,詹扬,彭忻烨,任小文,罗文钊。感谢你们在我准备博士论文及答辩期间的热心帮助。

感谢在华师读研以来风雨相伴的几位同窗好友:龚晖,陈蔚,刘可,饶识,吴妍,谷文举,伊珍,李延芳,曾世碧,朱剑辉,罗覃,
张瑛,张煜,鄢君。有你们在的日子,从来就不会缺少欢乐,感谢你们陪我走过这一段难忘的岁月。

感谢柯宏伟,仇浩和杜成民在我初到BNL时对我的悉心帮助。在异国他乡的不适应中,你们教会了我做饭的生存技能,
帮助我在一片浑浑噩噩的状态中安顿下来。感谢陈丽珠,朱逾卉,崔相利,黄炳矗,赵杰,薛亮,李玄,杨岩,辛科峰,韩立欣等几位师兄师姐在学业上
和生活上给我的莫大帮助。感谢查王妹,寿齐烨,阎威华,杨驰,郭毅,胡雪野,徐亦飞,杨帅,张金龙,张正桥,王旭,杨钱,周龙,黄欣杰,冯照
等小伙伴在BNL的朝夕相伴。感谢邓建老师,邵明老师,唐泽波老师,张一飞老师,王亚平老师,裴骅老师,施梳苏老师和王晓蓉老师在BNL期间对我的照顾。

感谢刘嘉师兄和郝阳师姐在我们的英语学习小组中对我的帮助。和你们一起经历的挑战让我觉得这世上没有不能克服的困难。


感谢亲爱的父亲母亲一直以来对我的支持。我能理解很多时候你们所承受的压力比我还要大,
所以我非常感激你们愿意去支持我所做出的每一个决定。


当五年多的时光化作手中这本薄薄的毕业论文,很难想象一段超过二十分之一的人生旅程就要在这里做出一个小小的总结。
重新翻过自己的论文,好像看到的都是这五年来的点点滴滴:有在处理未知问题时的忐忑不安,有在寻求解决方案时的辗
转反侧,有在完成一个小任务时的暗自得意,也有在看到新方向时的兴奋不已。翻过这一页,是道别,也是启程。感谢那
些陪我一路走来的人们,你们让我前进的每一步都有了存在的意义。