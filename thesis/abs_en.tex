\chapter*{\LARGE \bfseries {Abstract}}

\normalsize { 

Nuclear physics is very concerned about the emergence of the atom, the nucleus
and the nucleons within it, accounting essentially for the visible mass of our
most immediate universe. Studies over the past half a century have revealed that
the nucleons are consisted of quarks in a bounded state through an interaction
based on the exchange of gluons. The quantum chromodynamics (QCD) is founded to
describe and improve our understanding to this interaction based on the
leapfrogging development in related experimental and theoretical fields. It is
the ultimate goal of nuclear physics to understand the nuclear structure from
the dynamics of quarks and gluons within QCD framework.

The interaction in QCD is attributed to the color charge taken by the quarks
and gluons. While gluons are force carrier, they can take color charge themselves.
Due to this unique feature, gluon contribution is expected to be strong in the
formation of nucleon structure. The inner structure of a nucleon can be explored
by smashing it into smaller pieces. The scattering over a nucleon can be interpreted
as the incoherent superposition of scatterings on the fundamental constituent
quarks and gluons, leading us to a universal parton distribution function (PDF)
describing the densities for quarks and gluons in a nucleon. Although past
experiments were successful in determining the quark behavior in the nucleon and
light nuclei, the gluons that determine the essential features of the strong
interactions, remain largely unexplored. Of great interest is especially the
high parton density (small $x$) regime where gluon self-interaction is expected to
dominate and lead to parton saturation. 

The current theoretical impulses suggest that this saturation effect may come
from the nonlinear evolution for gluons. We have got some tantalizing hints
for the existence of such a feature in the current experimental searches, while
no conclusive evidence is obtained. The proposed electron-ion collider (EIC) with
the possibility to collide multiple nuclear beams in a wide energy range is
believed to be an ideal experimental facility to provide us answers in a very high
precision to the search of saturation physics.

Two-particle azimuthal angle correlations have been reckoned to be one of the
most direct and sensitive probes to access the underlying gluon dynamics on the
future EIC. In this thesis, we will report our Monte Carlo studies targeted to
understand the feasibility of performing this dihadron correlation measurement
on an EIC. Additionally, a possible way to constrain the underlying
electron-nucleus collision geometry has been discussed. This method if achieved
can be very beneficial to precisely study of dihadron correlation measurement.
We have shown that it is very promising to carry out this study at the future
EIC experimental facility.


\vspace{4mm}


\textbf{Keywords:} electron-ion collider(EIC), electron-nucleus collision, QCD, gluon saturation, dihadron correlation


}
